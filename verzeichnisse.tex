%Verzeichnisse

% vor dem Hauptteil römische Seitenzahlen verwenden
\pagenumbering{Roman}
\newpage
% dies ist die erste Seite bezüglich der Seitenzahlen
\setcounter{page}{1}

% Inhaltsverzeichnis
\tableofcontents
\newpage

% Abkürzungsverzeichnis
\phantomsection \addcontentsline{toc}{section}{Abkürzungsverzeichnis}
\section*{Abkürzungsverzeichnis}
\begin{acronym}[ECC 6.0] % für die Ausrichtung die längste Abk. hier in eckigen Klammern
% \acro{ERP}{Enterprise-Resource-Planning} 
% \acro{ETL}{Extract, Transform, Load} 
 \acro{CS}{Column-Store, spaltenorientierter Datenspeicher} 
 \acro{CSV}{Comma Separated Values, Plaintext-Datenformat} 
 \acro{DBMS}{Datenbankmanagementsystem} 
 \acro{DRAM}{Dynamischer RAM, 1 Transistor + 1 Kondensator pro Bit, langsamer als SRAM} 
 \acro{ECC}{SAP Enterprise Core Component} 
 \acro{HDD}{Hard Disk Drive, Laufwerk mit Magnetspeicher-Platten} 
 \acro{HCP}{HANA Cloud Platform} 
 \acro{OLAP}{Online Analytical Processing, Datenanalyse von Geschäftsdaten} 
 \acro{OLTP}{Online Transaction Processing, Online-Transaktionsverarbeitung} 
 \acro{RAM}{Random Access Memory, Arbeitsspeicher} 
 \acro{RS}{Row-Store, zeilenorientierter Datenspeicher} 
% \acro{SaaS}{Software as a Service, Software als Dienstleistung} 
 \acro{SSD}{Solid State Drive, Halbleiterlaufwerk (Flash-EPROM-Speicher)} 
 \acro{SRAM}{Statischer RAM, 6 Transistoren pro Bit, schneller und teurer als DRAM} 

\end{acronym}

% ggf. Abbildungsverzeichnis auf eine neue Seite
 \newpage 
 \addcontentsline{toc}{section}{Abbildungsverzeichnis}
 \listoffigures

% ggf. Tabellenverzeichnis
 \addcontentsline{toc}{section}{Tabellenverzeichnis}
 \listoftables

\newpage
\pagenumbering{arabic}
