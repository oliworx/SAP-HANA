% Literaturliste soll im Inhaltsverzeichnis auftauchen
\newpage
\addcontentsline{toc}{section}{Literatur}
  \begin{thebibliography}{}

%  Monographien:
%1.  Familienname des Verfassers
%2.  Vorname des Verfassers, vorzugsweise abgekürzt
%3.  Erscheinungsjahr in Klammern
%4.  vollständiger Titel des Werks
%5.  Auflage (Aufl.), wenn es sich nicht um die erste Auflage handelt
%6.  Erscheinungsort, der grundsätzlich dem Verlagsort entspricht (mehr als 3 Orte sind nicht zu nennen; stattdessen: u.a.) und / oder Verlag
%7.  Erscheinungsjahr
%8.  Handelt es sich bei der Literaturquelle um eine Dissertation, ist vor 5. die Abkürzung „Diss.“ Einzufügen

% Beispiele:
% Scholz, C. (2000): Personalmanagement, 5. Aufl., München 2000
% Lüdenbach, N., Hoffmann, W.-D. (2003): Haufe IAS-Kommentar, Freiburg 2003
% Ruhnke, K. (1995): Konzernbuchführung, Diss., Düsseldorf 1995

% Zeitschriftenartikel oder Zeitungsartikel:
% 1.  Familienname des Verfassers
% 2.  Vorname des Verfassers, vorzugsweise abgekürzt
% 3.  Erscheinungsjahr in Klammern
% 4. vollständiger Titel des Artikels
% 5. nach dem Titel wird der Zusatz „in:“ eingefügt
% 6. Name der Zeitung in abgekürzter oder ausgeschriebener Form mit Erscheinungsjahr
% 7. Jahrgang (Jg.)
% 8. ggf.Heftnummer (Nr.),wenn keine durchgängige Paginierung(Durchnummerierung) erfolgt
% 9. Seitenzahl („S. ...“) bzw. Spaltenzahl („Sp. ...“), und zwar Anfang bis Ende
%
% Beispiele:
% Herzberg, F. (1968): Was Mitarbeiter in Schwung bringt, in: Harvard Business Manager, 
%   o.J., 2003, Nr. 4, S. 50-62
% Kümpel, Th. (2002): Integration von internem und externem Rechnungswesen bei der Bewertung
%   erfolgversprechender langfristiger Fertigungsaufträge, in: Der Betrieb 2002, 55. Jg., S. 905-910

	% Hauptquelle
	\bibitem[Plattner, H., Leukert, B. (2015)]{Plattner2015} {\sl The In-Memory Revolution - How SAP HANA enables business of the future}, Springer, 2015
	\bibitem[Prassol, P. (2015)]]{Prassol2015} {\sl SAP HANA als Anwendungsplattform für Real-Time Business}, Springer Fachmedien Wiesbaden, 2015
	
	\textbf{Internetquellen:}
% 1. Nachname, ggf. Internetanbieter,
% 2. Vorname Jahr:
% 3. Titel.
% 4. URL: http://(Internetadresse),
% 5. Abruf am (Datum).
% Beispiel:
% Teltarif (2007): Rufnummerportierung im Mobilfunk. URL: http://www.teltarif.de/i/portierung.html, Abruf am 7.11.2007

	% bis 6 TB RAM
	\bibitem[Intel (2014)]{Intel2014}{\sl Scaling Data Capacity for SAP HANA with Fujitsu Servers} URL: \url{http://www.intel.de/content/www/de/de/big-data/big-data-xeon-e7-sap-hana-fujitsu-paper.html}, Abruf am 28.8.2016
	% bis 24 Kerne
	\bibitem[Intel (2016)]{Intel2016}{\sl Intel Xeon Processor E7-8890 v4 } URL: \url{http://ark.intel.com/de/products/93790/Intel-Xeon-Processor-E7-8890-v4-60M-Cache-2_20-GHz}, Abruf am 28.8.2016
        % In-Memory Data Management
	\bibitem[Plattner/HPI (2015)]{Plattner2015hpi}{\sl In-Memory Data Management 2015, Prof. Hasso Plattner} URL: \url{https://open.hpi.de/courses/imdb2015}, Abruf am 20.8.2016
        % scale-up / scale-out
	\bibitem[SAP (2014)]{SAP2014}{\sl SAP HANA SPS 09 - What’s New? SAP HANA Scalability} URL: \url{https://hcp.sap.com/content/dam/website/saphana/en_us/Technology%20Documents/SPS09/SAP%20HANA%20SPS09%20-%20HANA%20Scalability.pdf}, Abruf am 30.8.2016
	% bis zu 12 TB /96 cores pro System (S.5) Intel Xeon und IBM Power (S.9)
	\bibitem[SAP (2016)]{SAP2016}{\sl SAP HANA Tailored Data Center Integration - Overview} URL: \url{http://go.sap.com/documents/2016/05/827c26ba-717c-0010-82c7-eda71af511fa.html}, Abruf am 27.8.2016
	% Anzahl Orte in D
	\bibitem[Statista (2014)]{Statista2014}{\sl Anzahl der Gemeinden in Deutschland nach Gemeindegrößenklassen} URL: \url{http://de.statista.com/statistik/daten/studie/1254/umfrage/anzahl-der-gemeinden-in-deutschland-nach-gemeindegroessenklassen/}, Abruf am 27.8.2016
	 
  \end{thebibliography}
